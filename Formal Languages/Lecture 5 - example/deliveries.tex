\documentclass[11pt]{article}
\setlength{\topmargin}{13mm}
\setlength{\headheight}{0mm}
\setlength{\headsep}{0mm}
\setlength{\textheight}{225mm}
\setlength{\oddsidemargin}{0mm}
\setlength{\textwidth}{160mm}
\usepackage{supertabular}
\usepackage{amsfonts}
%\input{psfig}

\begin{document}
\begin{sloppypar}

% DEFINITION DES CARACTERES MATHEMATIQUES B
%------------------------------------------
\def\@setmcodes#1#2#3{{\count0=#1 \count1=#3
	\loop \global\mathcode\count0=\count1 \ifnum \count0<#2
	\advance\count0 by1 \advance\count1 by1 \repeat}}

\@setmcodes{`A}{`Z}{"7441}
\@setmcodes{`a}{`z}{"7461}

\mathcode`\;="8000 % Makes ; active in math mode
{\catcode`\;=\active \gdef;{\semicolon\;}}
\mathchardef\semicolon="003B
%    Nominal distance from top of paper to top of page
\topmargin 0 pt
\textheight 53\baselineskip

%   Left margin on odd-numbered pages
\oddsidemargin  0.15 in
%   Left margin on even-numbered pages
\evensidemargin 0.35 in
%   Width of marginal notes.
\marginparwidth 1 in
%   Note that \oddsidemargin = \evensidemargin
\oddsidemargin 0.25 in
\evensidemargin 0.25 in
\marginparwidth 0.75 in
\textwidth 5.875 in % Width of text line.

\setlength{\parindent}{0pt}
\setlength{\parskip}{0ex}

% DEFINITION DES FONTS
%---------------------
% The AMS extra symbol fonts are loaded.
% Note: sometimes called euxm10
\font\msx=msam10
% Note: sometimes called euym10
\font\msy=msbm10

\newfam\msxfam \textfont\msxfam=\msx
\newfam\msyfam \textfont\msyfam=\msy

\def\famletter#1{\ifcase #1 0\or 1\or 2\or 3\or 4\or 5\or 6\or 7\or
	8\or 9\or A\or B\or C\or D\or E\or F\fi}

\edef\fx{\famletter\msxfam}
\edef\fy{\famletter\msyfam}

\def\bbold{\fam\msyfam \msy}

% SYMBOLES B
%-----------
% makes a quoted expression in mathematical text
\def\token#1{\hbox{`$#1$'}}
% used for error messages in Z specs
\def\report#1{\hbox{`{\tt #1}'}}

% \@myop makes an operator, with a strut to defeat TeX's vertical adjustment.
\def\@myop#1{\mathop{\mathstrut{#1}}\nolimits}

% This underscore doesn't have the little kern --- you get an italic
% correction anyway in math mode.
\def\_{\leavevmode \vbox{\hrule width0.5em}}

% Save \q as \xq for quantifiers q.
\let\xforall=\forall
\let\xexists=\exists
\let\xlambda=\lambda
\let\xmu=\mu

% \p and \f make arrows with 1 and 2 crossings resp.
\def\p#1{\mathrel{\ooalign{\hfil$\mapstochar\mkern 5mu$\hfil\cr$#1$}}}
\def\f#1{\mathrel{\ooalign{\hfil
	$\mapstochar\mkern 3mu\mapstochar\mkern 5mu$\hfil\cr$#1$}}}

\let\mc=\mathchardef

\def	\pow		{\mbox{${\mathbb P}$}}
\def	\po1		{\mbox{${\cal P}_1$}}
\let	\cross		\times
\def	\lambda		{\@myop{\xlambda}}
\def	\lnot		{\neg\;}
\def	\land		{\mathrel{\wedge}}
\def	\lor		{\mathrel{\vee}}
\let	\implies	\Rightarrow
\let	\iff		\Leftrightarrow
\def	\forall		{\@myop{\xforall}}
\def	\exists		{\@myop{\xexists}}
\def	\semi		{\mathrel{\comp}}
\def	\ssemi		{\mathbin{\rm ;}}
\let	\ensembleVide	\emptyset
\let	\rel		\leftrightarrow
\def	\dom		{\@myop{\sf dom}}
\def	\ran		{\@myop{\sf ran}}
\def	\id		{\@myop{\sf id}}
\def	\comp		{\mathbin{\raise
			0.6ex\hbox{\oalign{\hfil$\scriptscriptstyle
			\rm o$\hfil\cr\hfil$\scriptscriptstyle\rm 9$\hfil}}}}
\def	\para		{\mbox{$\mid\mid$}}
\mc	\dres		"2\fx43
\mc	\rres		"2\fx42
\def	\ndres		{\mathbin{{\dres} \llap{$-$}}}
\def	\nrres		{\mathbin{{\rres}\llap{$-$}}}
\def	\lover		{\mathbin{{\dres} \llap{$-\!\!\!\!-\!$}}}
\def	\rover		{\mathbin{{\rres}\llap{$\!-\!\!\!-$}}}
\let	\fun		\rightarrow
\def	\pfun		{\p\fun}
\def	\pinj		{\p\inj}
\mc	\inj		"3\fx1A
\def	\psurj		{\p\surj}
\mc	\surj		"3\fx10
\def	\bij		{\surj\!\!\!\!\!\!\!\inj}
\def	\nat		{\mbox{${\cal N}$}}
\def	\na1		{\mbox{${\cal N}_1$}}
\def	\num		{\mbox{${\cal Z}$}}
\def	\int		{\mbox{${\cal Z}$}}
\def	\rat		{\mbox{${\cal Q}$}}
\def	\div		{\mathbin{\rm /}}
\def	\mod		{\mathbin{\bf mod}}
\def	\upto		{\mathbin{\ldotp\ldotp}}
\def	\finset		{\mbox{${\cal F}$}}
\def	\finse1		{\mbox{${\cal F}_1$}}
\def	\ffun		{\f\fun}
\def	\finj		{\f\inj}
\def	\seq		{\@myop{\rm seq}}
\def	\cat		{\mathbin{\raise 0.8ex\hbox{$\mathchar"2\fx61$}}}
\def	\sep		{\hspace*{.05in}}

\setcounter{secnumdepth}{0}
\setcounter{tocdepth}{0}

%-------------------%
% Debut du document %
%-------------------%

\textbf{Problem 2}: Give a machine that captures the following description and check its consistency. \\[0.2cm]

A \textit{Deliveries} machine keeps track of the items on a delivery van and the addresses to which they should be delivered. It also keeps track of a special set of addresses \textit{nogo} for which there might be problems in making deliveries. \\
Initially, the van is empty and the set \textit{nogo} can be initialized to any arbitrary set of addresses. \\
The machine provides four operations:
\begin{enumerate}
  \item \textbf{load} takes an item \textit{ii} and an address \textit{aa} as input and adds \textit{ii}, to be delivered to \textit{aa}, to the 
	       contents of the van.
	\item \textbf{drop} should only be invoked when the van is not empty. In such a case, it chooses an arbitrary item \textit{ii} from the van and 
	      delivers it to address \textit{aa}; these two values are provided as outputs of the operation.			
	\item \textbf{endofday} can always be invoked. It nondeterministically chooses either to empty the van, or to leave it as it is. It has no 
	      inputs or outputs. 
	\item \textbf{warning} takes an address \textit{aa} as input. If the address is in \textit{nogo}, then it might remove all items associated to 
	      that address from the van; or, alternatively, it might remove the address from \textit{nogo}. If the address is not in \textit{nogo} and  
				there are no deliveries to that address, then it will be inserted into \textit{nogo}. In all the other cases, the operation has no 
				effect. 			
\end{enumerate}

\end{sloppypar}\end{document}
