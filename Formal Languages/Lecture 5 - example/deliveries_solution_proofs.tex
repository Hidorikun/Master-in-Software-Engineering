\documentclass[11pt]{article}
\setlength{\topmargin}{13mm}
\setlength{\headheight}{0mm}
\setlength{\headsep}{0mm}
\setlength{\textheight}{225mm}
\setlength{\oddsidemargin}{0mm}
\setlength{\textwidth}{160mm}
\usepackage{supertabular}
\usepackage{amsfonts}
\usepackage{amsmath}
%\input{psfig}

\begin{document}


The general patterns for the proof obligations to discharge for showing the consistency of a simple abstract machine are given at the end of Lecture 4. We abstract away from the contextual abbreviations. Since machine \texttt{Deliveries} has no assertions, constraints and properties, proving its consistency involves:
\begin{enumerate}
  \item proving that the initialization establishes the invariant, namely $[U]I$
	\item proving that each operation preserves it, namely  $I \wedge Q \Rightarrow [V]I$,
\end{enumerate}
where
\begin{align}
I = \; & items \subseteq ITEM \; \wedge\\
       & deliveries \in items \rightarrow ADDRESS \; \wedge \\
			 & nogo \subseteq ADRESS
\end{align}


1. \textbf{Initialization}
  \begin{align}
     [U]I = \; & [items := \phi \; || \; deliveries := \phi \; || \; nogo :\in \mathbb{P}(ADDRESS)] \; I = \nonumber \\	  
		      = \; & [items := \phi \; || \; deliveries := \phi \; || \; \nonumber \\
					  \; & || \; ANY \; s \; WHERE \; s \in \mathbb{P}(ADDRESS) \;\; THEN \; nogo := s \; END] \; I = \nonumber \\	  
					= \; & [\; ANY \; s \; WHERE \; s \in \mathbb{P}(ADDRESS)  \nonumber \\
					  \; & \;\; THEN \; items := \phi \; || \; deliveries := \phi \; || \; nogo := s \; END] \; I = \nonumber \\
					= \; & \forall \; s \; . \; ( s \in \mathbb{P}(ADDRESS)\Rightarrow [items := \phi \; || \; deliveries := \phi \; || \; nogo := s] \; I )             =	\nonumber  \\
          = \; & \forall \; s \; . \; ( s \in \mathbb{P}(ADDRESS) \Rightarrow \phi \subseteq ITEM \; \wedge \phi : \phi \rightarrow ADDRESS \;              \wedge s \subseteq ADRESS )
   \end{align}

Predicate (4) is obviously true \\

2. \textbf{Operations} \\

2.1 Operation \texttt{load}

We have 
\begin{align}
  Q = \; & ii \in ITEM - items \; \wedge\\
	       & aa \in ADDRESS \\
	V = \; & items := items \cup \{ii\} \; || \; deliveries(ii) := aa \nonumber
\end{align}

Then
\begin{align}
    [V]I = \; & [items := items \cup \{ii\} \; || \; deliveries(ii) := aa] \; I = \nonumber \\	  
			   = \; & items \cup \{ii\} \subseteq ITEM \; \wedge \\
					    & deliveries \triangleleft \!\!\!\!- \{ ii \mapsto aa\} \in items \cup \{ii\} \rightarrow ADDRESS \; \wedge \\
							& nogo \subseteq ADDRESS
\end{align}

(9) is equivalent to (3), thus true

(7) follows from (1) and (5)

(8) follows from (2), (5) and (6) \\


2.2 Operation \texttt{drop}

We have 
\begin{align}
  Q = \; & items \neq \phi \\
	V = \; & ANY \; ii \; WHERE \; ii \; \in \; items \; THEN  \; items := items - \{ii\} \; || \nonumber \\
	       & || \; deliveries := \{ii\} \mathbin{\triangleleft \llap{-}} deliveries \; || \; it,ad := ii, deliveries(ii) \; END \nonumber
\end{align}

Let us denote
\begin{align*}
  V' = & \; items := items - \{ii\} \; || \; deliveries := \{ii\} \mathbin{\triangleleft \llap{-}} deliveries \; || \\
	     & \; || \; it,ad := ii, deliveries(ii)
\end{align*}

Then
\begin{align}
    [V]I = & \; [ANY \; ii \; WHERE \; ii \in items \; THEN \; V' \; END ] \; I  = \nonumber \\
		     = & \; \forall \; ii \; . \; (ii \in items \Rightarrow [V']I ) \nonumber 
\end{align}
\begin{align}
    [V']I = \; & items - \{ii\} \subseteq ITEM \; \wedge \\
		        \; & \{ii\} \mathbin{\triangleleft \llap{-}} deliveries \in items - \{ii\} \rightarrow ADDRESS \; \wedge \\
						\; & nogo \subseteq ADDRESS
\end{align}

(13) is equivalent to (3), thus true for any $ii \in items$.

(11) follows from (1) for any $ii \in items$.

(12) follows from (2) for any $ii \in items$. \\


2.3 Operation \texttt{endofday} 

We have 
\begin{align}
  Q = \; & true \nonumber \\
	V = \; & CHOICE \; items,deliveries := \phi, \phi \; OR \; skip \; END \nonumber
\end{align}

Then
\begin{align}
    [V]I = & \; [CHOICE \; items,deliveries := \phi, \phi \; OR \; skip \; END] \; I \nonumber = \\
		     = & \; [items,deliveries := \phi, \phi]\;I \wedge [skip]\;I = \nonumber \\
				 = & \; [items,deliveries := \phi, \phi]\;I \wedge I
\end{align}

The first conjunct of the predicate (14) is provable by analogy to the similar part from the proof concerning the initialisation. \\


2.4 Operation \texttt{warning} 

We have 
\begin{align}
  Q = \; & aa \in ADDRESS  \\
	V = \; &  IF \; aa \in nogo \; THEN \; V_1 \; ELSE \; V_2 \; END \nonumber \\
	V_1 = \; & CHOICE \; nogo := nogo - \{aa\} \; OR \nonumber \\
	         & deliveries := deliveries \mathbin{\triangleright \llap{--}} \{aa\} \; || \; items := items - deliveries^{-1}[\{aa\}] \; END\nonumber  \\
	V_2	= \; & IF \; aa \notin ran(deliveries) \; THEN \; nogo := nogo \cup \{aa\} \; END \nonumber			
\end{align}

Then
\begin{align}
  [V] I = & \; (aa \in nogo \Rightarrow [V_1]I) \wedge (\neg \; aa \in nogo \Rightarrow [V_2]I)
\end{align}
\begin{align}
  [V_1]I = \; & [nogo := nogo - \{aa\}] \; I \; \wedge \\
	            & [deliveries := deliveries \mathbin{\triangleright \llap{--}}
 \{aa\} \; || \; items := items - deliveries^{-1}[\{aa\}]] \; I
\end{align}

(17) reduces to $nogo - \{aa\} \subseteq ADDRESS$, which is true based on (3). \\

Proving (18) means proving that
\begin{align}
  & items - deliveries^{-1}[\{aa\}] \subseteq ITEM \; \wedge \\
	& deliveries \mathbin{\triangleright \llap{--}}
 \{aa\} \in items - deliveries^{-1}[\{aa\}] \rightarrow ADDRESS
\end{align}

(19) follows from (1) and (15).

(20) follows from (2) and (15).

\begin{align}
 [V _2]I = & \; (aa \notin ran(deliveries) \Rightarrow [nogo := nogo \cup \{aa\}] I ) \; \wedge \\
	         & \; (\neg aa \notin ran(deliveries) \Rightarrow [skip] I)
\end{align}

(22) is obviously true, while (21) reduces to $nogo \cup \{aa\} \subseteq ADDRESS$, which is true based on (3) and (15).


\end{document}
